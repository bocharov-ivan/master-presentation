\documentclass{beamer}
%математика
\usepackage{amssymb,amsmath,mathtext}
\usepackage{indentfirst,amsfonts}
\usepackage{makecell,multirow,longtable}
%язык
\usepackage[english,russian]{babel}
\usepackage[utf8]{inputenc}
% Стиль презентации
\usetheme{Warsaw}
\begin{document}
\title[Линейные и нелинейные модели]{Линейные и нелинейные модели в задачах автоматической классификации текстов на естественных языках}  
\author[Бочаров И.А., А-13-08]{Бочаров И.А., А-13-08 \\Научный руководитель: д.т.н., проф. Фальк В.Н.,\\Консультант: Шаграев А.Г.}

\institute{НИУ МЭИ, АВТИ, Кафедра Прикладной математики}
\date{Москва, 2014} 
% Создание заглавной страницы
\begin{frame}[plain]
	\titlepage
\end{frame}
% Автоматическая генерация содержания
\begin{frame}
	\tableofcontents
\end{frame}


\section{Постановка задачи классификации}
\begin{frame}
\frametitle{Постановка задачи классификации}
Постановка $$\sqrt{f_3}$$

\end{frame}

\section{Stuff1}
\begin{frame}
\frametitle{}
\begin{itemize}
\item Раз.
\item Два.
\item Три.
\end{itemize}
\end{frame}

\section{Stuff2}
\begin{frame}
\frametitle{}
\begin{itemize}
\item Раз.
\item Два.
\item Три.
\end{itemize}
\end{frame}


\end{document}
