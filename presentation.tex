\documentclass{beamer}
%математика
\usepackage{amssymb,amsmath,mathtext}
\usepackage{indentfirst,amsfonts}
\usepackage{makecell,multirow,longtable}
%язык
\usepackage[english,russian]{babel}
\usepackage[utf8]{inputenc}
% Стиль презентации
\usetheme{Warsaw}
\begin{document}
\title[Линейные и нелинейные модели]{Линейные и нелинейные модели в задачах автоматической классификации текстов на естественных языках}  
\author[Бочаров И.А., А-13-08]{Бочаров И.А., А-13-08 \\Научный руководитель: д.т.н., проф. Фальк В.Н.,\\Консультант: Шаграев А.Г.}

\institute{НИУ МЭИ, АВТИ, Кафедра Прикладной математики}
\date{Москва, 2014} 
% Создание заглавной страницы
\begin{frame}[plain]
	\titlepage
\end{frame}
% Автоматическая генерация содержания
\begin{frame}
	\tableofcontents
\end{frame}


\section{Постановка задачи классификации}
\begin{frame}
\frametitle{Постановка задачи классификации}
Пусть $X$ - множество классифицируемых объектов, а $Y$ - конечное множество классов. Имеется целевая функция - отображение $y^*:X\rightarrow Y$, значения которой известны только на конечном подмножестве объектов $X' \subset X$ обучающей выборки \\
$$X^l=\{\langle x_i,y_i \rangle |x_i \in X', y_i=y^*(x_i),1\le i \le l\}$$
Необходимо построить решающую функцию $a: X \rightarrow Y$ , принадлежащую
некоторому классу функций $\theta$, которая была бы как можно более качественным
приближением к целевой функции. 
\\Методом обучения $\mu$ будем называть функцию, ставящую в соответствие любой обучающей выборке некоторую
решающую функцию из класса $\theta: \mu(X^l)=a \in \theta$
\end{frame}

\end{document}
