\documentclass{beamer}
%математика
\usepackage{amssymb,amsmath,mathtext}
\usepackage{indentfirst,amsfonts}
\usepackage{makecell,multirow,longtable}
%язык
\usepackage[english,russian]{babel}
\usepackage[utf8]{inputenc}
% Стиль презентации
\usetheme{Warsaw}
\useoutertheme{infolines}
\begin{document}
\title[Линейные и нелинейные модели]{Линейные и нелинейные модели в задачах автоматической классификации текстов на естественных языках}  
\author[Бочаров И.А.]{Бочаров И.А., А-13-08 \\Научный руководитель: д.т.н., проф. Фальк В.Н.,\\Консультант: Шаграев А.Г.}

\institute{НИУ МЭИ, АВТИ, Кафедра Прикладной математики}
\date{Москва, 2014} 
% Создание заглавной страницы
\begin{frame}[plain]
	\titlepage
\end{frame}
% Автоматическая генерация содержания
\begin{frame}
	\tableofcontents
\end{frame}


\section{Постановка задачи классификации}
\begin{frame}
\frametitle{Постановка задачи классификации}
Пусть $X$ - множество классифицируемых объектов, а $Y$ - конечное множество классов. Имеется целевая функция - отображение $y^*:X\rightarrow Y$, значения которой известны только на конечном подмножестве объектов $X' \subset X$ обучающей выборки \\
$$X^l=\{\langle x_i,y_i \rangle |x_i \in X', y_i=y^*(x_i),1\le i \le l\}$$
Необходимо построить решающую функцию $a: X \rightarrow Y$ , принадлежащую
некоторому классу функций $\theta$, которая была бы как можно более качественным
приближением к целевой функции. 
\\Методом обучения $\mu$ будем называть функцию, ставящую в соответствие любой обучающей выборке некоторую
решающую функцию из класса $\theta: \mu(X^l)=a \in \theta$
\end{frame}

\section{Оценка качества классификации}
\begin{frame}
\frametitle{Оценка качества классификации}
Для измерения качества предсказаний необходимо определить функционал
качества – функцию, которая всякому набору прецедентов (пар, состоящих из
объектов и соответствующих им ответов) и решающей функции сопоставляет
некоторое число. Считается, что, чем больше значение функционала качества, тем лучше качество предсказаний решающей функции.
\newline
\newline
\newline
Пусть $X^t$-тестовая выборка. Тогда значение функционала $$Q(X^t,\mu(X^l))$$ можно считать оценкой обобщающей способности метода обучения $\mu$.
\end{frame}

\subsection{Скользящий контроль}
\begin{frame}
\frametitle{Скользящий контроль}
	Пусть имеется обучающая выборка $X^l$, метод обучения $\mu$ и некоторый функционал качества $Q$. Зафиксируем множество разбиений выборки $X^l$:
	$$\{\{S_1^l,S_1^t\},\{S_2^l,S_2^t\},...,\{S_k^l,S_k^t\}\},$$
	$$S_i^l \cup S_i^t = X^l, i=1..k$$
	Оценка качества, усредненная по всем разбиениям, называется оценкой скользящего контроля:
	$$CV(Q,\mu,S)=\frac{1}{k}\sum\limits_{i=1}^{k}Q(S_i^t,\mu(S_i^l))$$
\end{frame}

\begin{frame}
\frametitle{Стратификация выборки}
	Желательно, чтобы разбиения, используемые при получении оценок методом скользящего контроля обладали теми же статистическими свойствами, что и вся выборка $X^l$. Для достижения этого часто используется стратификация.
	\newline
	\newline
	Стратификация классов заключается в том, что разбиения проводятся таким образом, что доля каждого класса $k$ в каждой подвыборке $X_i^l$ примерно равна:
	$$\frac{|X_i^l|}{|X^l|}\sum\limits_{\langle x,y \rangle \in X^l}^{}[y=k],$$
	где:\\
	\begin{equation*}
	[X]=
	\begin{cases}
	&1, если\ условие\ X\ выполняется,\\
	&0, в\ противном\ случае.
	\end{cases}
	\end{equation*}
\end{frame}

\section{}
\begin{frame}
\frametitle{Спасибо за внимание!}
	Спасибо за внимание!
\end{frame}

\end{document}
